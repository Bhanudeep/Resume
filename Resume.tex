%-------------------------
% Rezume, a latex resume template for developers
% Author : Nanu Panchamurthy
% Based off of: https://github.com/sb2nov/resume
% License : MIT

% Hope this resume template helps you land an awesome job. If you found this helpful, please consider starring the github repo here, .
%-------------------------



%------------PACKAGES----------------
\documentclass[a4paper,11pt]{article}

\usepackage{verbatim} % reimplements the "verbatim" and "verbatim*" environments
\usepackage{xcolor}
\usepackage{sectsty}

\usepackage{titlesec} % provides an interface to sectioning commands i.e. custom elements

\usepackage{color} % provides both foreground and background color management

\usepackage{enumitem} % provides control over enumerate, itemize and description

\usepackage{fancyhdr} % provides extensive facilities for constructing headers, footers and also controlling their use

\usepackage{tabularx} % defines an environment tabularx, extension of "tabular" with an extra designator x, paragraph like column whose width automatically expands to fill the width of the environment

\usepackage{latexsym} % provides mathematical symbols

\usepackage{marvosym} % provides martin vogel's symbol font which contains various symbols

\usepackage[empty]{fullpage} % sets margins to one inch and removes headers, footers etc..

\usepackage[hidelinks]{hyperref} % removes color and shadow of hyperlinks

\usepackage[normalem]{ulem} % provides "\ul" (uline) command which will break at line breaks

\usepackage[english]{babel} % provides culturally determined typographical rules for wide range of languages
%-----------------------------------------

\input glyphtounicode % converts glyph names to unicode
\pdfgentounicode=1 % ensures pdfs generated are ats readable

%----------FONT OPTIONS-------------------
\usepackage[default]{sourcesanspro} % uses the font source sans pro
\urlstyle{same} % changes url font from default urlfont to font being used by the document
%-----------------------------------------


%----------MARGIN OPTIONS-----------------
\pagestyle{fancy} % set page style to one configured by fancyhdr
\fancyhf{} % clear all header and footer fields

\renewcommand{\headrulewidth}{0in} % sets thickness of linerule under header to zero
\renewcommand{\footrulewidth}{0in} % sets thickness of linerule over footer to zero

\setlength{\tabcolsep}{0in} % sets thickness of column separator in tables to zero

% origin of the document is one inch from the top and from and the left
% oddsidemargin and evensidemargin both refer to the left margin
% right margin is indirectly set using oddsidemargin
\addtolength{\oddsidemargin}{-0.5in}
\addtolength{\topmargin}{-0.5in}

\addtolength{\textwidth}{1.0in} % sets width of text area in the page to one inch
\addtolength{\textheight}{1.0in} % sets height of text area in the page to one inch

\raggedbottom{} % makes all pages the height of current page, no extra vertical space added
\raggedright{} % makes all pages the width of current page, no extra horizontal space added
%------------------------------------------


%--------SECTIONING COMMANDS---------
% \titleformat{<command>}
%   [<shape>]{<format>}{<label>}{<sep>}
%     {<before-code>}[<after-code>]

% command is the sectioning command to be redefined
% shape is the style of the font; scshape stands for small caps style
% format is the format to be applied to whole title- label and text; absent here
% label defines the label
% sep is the horizontal separation between label and title body
% before-code is the code to be executed before
% after-code is the code to be executed after

\titleformat{\section}
  {\scshape\large}{}
    {0em}{\color{blue}}[\color{black}\titlerule\vspace{0pt}]
%-------------------------------------


%--------REDEFINITIONS----------------
% redefines the style of the bullet point
\renewcommand\labelitemii{$\vcenter{\hbox{\tiny$\bullet$}}$}

% redefines the underline depth to 2pt
\renewcommand{\ULdepth}{2pt}
%-------------------------------------


%--------CUSTOM COMMANDS--------------
%\vspace{} defines a vertical space of given size, modifying this in custom commands can help stretch or shrink resume to remove or add content

% resumeItem renders a bullet point
\newcommand{\resumeItem}[1]{
  \item\small{#1}
}

% commands to start and end itemization of resumeItem, rightmargin set to 0.11in to avoid the overflow of resumetItem beyond whatever resumeItemHeading is being used
\newcommand{\resumeItemListStart}{\begin{itemize}[rightmargin=0.11in]}
\newcommand{\resumeItemListEnd}{\end{itemize}}

% resumeSectionType renders a bolded type to be used under a section, used as skill type here, middle element is used to keep ":"s in the same vertical line
\newcommand{\resumeSectionType}[3]{
  \item\begin{tabular*}{0.96\textwidth}[t]{
    p{0.15\linewidth}p{0.02\linewidth}p{0.81\linewidth}
  }
    \textbf{#1} & #2 & #3
  \end{tabular*}\vspace{-2pt}
}

% resumeTrioHeading renders three elements in three columns with second element being italicized and first element bolded, can be used for projects with three elements
\newcommand{\resumeTrioHeading}[3]{
  \item\small{
    \begin{tabular*}{0.96\textwidth}[t]{
      l@{\extracolsep{\fill}}c@{\extracolsep{\fill}}r
    }
      \textbf{#1} & \textit{#2} & #3
    \end{tabular*}
  }
}

% resumeQuadHeading renders four elements in a two columns with the second row being italicized and first element of first row bolded, can be used for experience and projects with four elements
\newcommand{\resumeQuadHeading}[4]{
  \item
  \begin{tabular*}{0.96\textwidth}[t]{l@{\extracolsep{\fill}}r}
    \textbf{#1} & #2 \\
    \textit{\small#3} & \textit{\small #4} \\
  \end{tabular*}
}

% resumeQuadHeadingChild renders the second row of resumeQuadHeading, can be used for experience if different roles in the same company need to added
\newcommand{\resumeQuadHeadingChild}[2]{
  \item
  \begin{tabular*}{0.96\textwidth}[t]{l@{\extracolsep{\fill}}r}
    \textbf{\small#1} & {\small#2} \\
  \end{tabular*}
}

% commands to start and end itemization of resumeQuadHeading, lefmargin for left indent of 0.15in for resumeItems
\newcommand{\resumeHeadingListStart}{
  \begin{itemize}[leftmargin=0.15in, label={}]
}
\newcommand{\resumeHeadingListEnd}{\end{itemize}}
%-------------------------------------------

\definecolor{amaranth}{rgb}{0.9, 0.17, 0.31}
%__________________RESUME____________________
% You can rearrange sections in any order you may prefer
\begin{document}

%-----------CONTACT DETAILS------------------
% Make sure all the details are correct, you can add more links in the first row of second column if needed

\begin{tabular*}{\textwidth}{l@{\extracolsep{\fill}}r}
  \textbf{\Huge Bhanudeep Simhadri \vspace{2pt}} & % row = 1, col = 1
  Location: Hyderabad, Telangana, India \\ % row = 1, col = 2
  \href{https://www.linkedin.com/in/bhanudeepsimhadri/}{\uline{LinkedIn}} $|$ % row = 2, col = 1
  \href{https://github.com/Bhanudeep}{\uline{GitHub}} $|$ % row = 2, col = 1
  \href{https://stackoverflow.com/users/19791101/bhanudeep-simhadri}{\uline{StackOverflow}} & % row = 2, col = 1
  Email: \href{mailto:bhanudeepsimhadry@icloud.com}{\uline{bhanudeepsimhadry@icloud.com}} $|$ % row = 2, col = 2
  Mobile: +91 7013695375 \\ % row = 2, col = 2
\end{tabular*}
%--------------------------------------------


%-----------SUMMARY--------------------------
% Keep this short, simple and straigth to point

\section{\color{amaranth} Software Developer}
\small{
  I am a highly skilled software developer experienced in \textbf{ C, Python, Flask, HTML, CSS, JavaScript}. I am looking for a position in the Computer Science industry where I can use and develop my skills while also being resourceful and adaptable to the growth of the organization and myself.
}
%--------------------------------------------


%--------------SKILLS------------------------
% Add or remove resumeSectionTypes according to your needs

\section{\color{amaranth} Technical Skills}
  \resumeHeadingListStart{}
    \resumeSectionType{Languages}{:}{Python, C, C++, JavaScript, HTML, CSS}
    \resumeSectionType{Frameworks}{:}{Flask }
    \resumeSectionType{Libraries}{:}{Tensorflow, face-recognition.api, Mininet, keras-ocr}
    \resumeSectionType{Databases}{:}{MongoDB, MySQL}
    \resumeSectionType{Dev Tools}{:}{Visual Studio Code, Git, Colab}
    \resumeSectionType{DataScience Tools}   {:}{RapidMiner}
  \resumeHeadingListEnd{}
%--------------------------------------------


%-----------EXPERIENCE-----------------------
% Distill all your talking points to small bullet points which follow the pattern "challenge-action-result" for maximum efficiency. Try to quantify (use numbers) your points whenver possible, highlist words of importance

\section{\color{amaranth} Experience}
\resumeHeadingListStart{}
\resumeQuadHeading{Full Stack Application Developer}{Aug 2022 -- Jan 2023}
  {Jawaharlal Nehru Technological University Hyderabad}{Remote -- Hyderabad, Telangana, India}
    \resumeItemListStart{}
      \resumeItem{Developed a complete desktop application for a system of attendance that uses face recognition. Focuses on simultaneously capturing attendance for a group of students in a class}
      \resumeItem{\textbf{Flask} framework was used in backend to build this application}
      \resumeItem{This web application was later \textbf{compiled to an executable file with all the necessary dependencies}. And is compatible with \textbf{Windows 10 or later} OS}
      \resumeItem{As the state government has mandated biometric-based attendance, This application is currently being used at various colleges. \href{https://github.com/Bhanudeep/Full-Stack-Desktop-Application-for-a-System-Based-on-Face-Recognition-for-Attendance}{\uline{Link to Application}}}
    \resumeItemListEnd{}
  \resumeQuadHeading{Software Developer Intern}{Sep2021 -- Jan 2022}
  {Cloud QA}{Remote -- Hyderabad, Telangana, India}
    \resumeItemListStart{}
      \resumeItem{Assisted senior web developers in developing and managing dynamic and responsive website using \textbf{HTML, CSS, JavaScript, and C\#}}
      \resumeItem{Worked with \textbf{Selenium} to automate elements in a webpage}
      \resumeItem{Fine tuned \textbf{website performance} and speed through optimization techniques}
    \resumeItemListEnd{}

  

\resumeHeadingListEnd{}
%---------------------------------------------


%-----------EDUCATION-------------------------
% Mention your CGPA, if its good, in the first row of second column

\section{\color{amaranth} Education}
  \resumeHeadingListStart{}
    \resumeQuadHeading{Jawaharlal Nehru Technological University Hyderabad}{Hyderabad, Telangana, India}
    {Bachelor of Technology in Computer Science and Engineering}{July 2018 -- July 2022}
  \resumeHeadingListEnd{}
%---------------------------------------------


%-----------PROJECTS--------------------------
% Use resumeQuadHeading if four elements are feasible (ex: demo video link), else use resumeTrioHeading. Keep the bullet points simple and concise and try to cover wide variety of skills you have used to build these projects

\section{\color{amaranth} Projects}
  \resumeHeadingListStart{}
    \resumeTrioHeading{\href{https://github.com/Bhanudeep/Detection-of-DDoS-attacks-on-SDN-network-using-Machine-Learning-}{\uline{Detection of DDoS attacks on SDN  network using Machine Learning}}}{Python, Mininet, hping3, iperf }{\href{https://github.com/Bhanudeep/Detection-of-DDoS-attacks-on-SDN-network-using-Machine-Learning-/}{\uline{Source Code}}}
      \resumeItemListStart{}
      \resumeItem{Designed and deployed local SDN network using \textbf{mininet}. Tools such as hping3 , iperf are used to generate DDoS and Normal traffic}
        \resumeItem{Designed and developed various Machine Learning models using \textbf{RapidMiner} to perform comparitive analysis on accuracy of various Machine Learning models using this \textbf{locally generated dataset}}
        \resumeItem{Best performing model among them was selected and deployed on the SDN network to monitor and detect DDoS network traffic}
      \resumeItemListEnd{}

      \resumeTrioHeading{\href{https://github.com/Bhanudeep/Accident-Detection-and-Rescue-System-using-Deep-Learning}{\uline{Accident Detection and rescue system using Deep Learning}}}{Python, Tensorflow, mongodb, SMTP}{\href{https://github.com/Bhanudeep/Accident-Detection-and-Rescue-System-using-Deep-Learning}{\uline{Source Code}}}
      \resumeItemListStart{}
        \resumeItem{A \textbf{Deep Learning} appliaction to detect accidents and alert emergengy services as well as victim's trustee}
        \resumeItem{A \textbf{Deep Learning} model was trained using \textbf{custom generated and labelled dataset} using \textbf{Tensorflow in google Colab} to detect accident} \resumeItem{When an accident is detected,\textbf{licence plate number} is extraced using \textbf{keras-ocr} and then details linked with this number are fetched, finally an alert is sent to victim's trustee and emergency services}
      \resumeItemListEnd{}
  \resumeHeadingListEnd{}
%--------------------------------------------


%----------------OTHERS----------------------
% You can add your acheivements, accolades, certifications etc. here.

%\section{Certifications}
%  \resumeItemListStart{}
%    \resumeItem{\href{https://dummy-certification.com}{\uline{Certified Web Developer by the W3C}}}
 %   \resumeItem{\href{https://dummy-certification.com}{\uline{Microsoft Certified: Azure Developer Associate}}}
 %   \resumeItem{\href{https://dummy-certification.com}{\uline{AWS Certified Developer - Associate}}}
%  \resumeItemListEnd{}
%--------------------------------------------

\end{document}